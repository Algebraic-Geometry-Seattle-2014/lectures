\documentclass{amsart}

\usepackage[utf8]{inputenc}
\usepackage[T1]{fontenc}
\usepackage{eulervm}
\usepackage{tgpagella}
\usepackage[all,arc]{xy}

\usepackage{amsmath}
\usepackage{amssymb}
\usepackage{amsthm}
\usepackage{amsfonts}
\usepackage{mathrsfs}
\usepackage{xspace}
\usepackage{tikz}
\usepackage{nicefrac}
\usepackage{fixmath}
\usepackage{paralist}
\usepackage{ellipsis}
\usepackage{mathtools}
\usepackage{graphicx,subfigure,epic,eepic}
\usepackage{array}
\usepackage{tikz-cd}

\usepackage{fixltx2e}
\usepackage[expansion=false,final]{microtype}

\newtheorem*{thm}{Theorem}
\newtheorem*{lem}{Lemma}
\newtheorem*{defn}{Definition}
\newtheorem*{prop}{Proposition}
\newtheorem*{cor}{Corollary}
\newtheorem*{ex}{Example}
\newtheorem*{rem}{Remark}
\newtheorem*{warn}{Warning}

\newcommand{\A}{\mathcal{A}}
\newcommand{\B}{\mathcal{B}}
\newcommand{\CC}{\textbf{C}}
\newcommand{\PP}{\textbf{P}}
\newcommand{\Oh}{\mathcal{O}}
\renewcommand{\H}{\mathcal{H}}

\DeclareMathOperator{\Bl}{Bl}
\DeclareMathOperator{\Coh}{Coh}
\DeclareMathOperator{\Ext}{Ext}
\DeclareMathOperator{\Hom}{Hom}
\DeclareMathOperator{\pt}{pt}
\DeclareMathOperator{\Spec}{Spec}



\title{Semiorthogonal Decomposition}
\author{John Calabrese, Notes by Brian Hwang}
\date{August 7, 2014}

\begin{document}

\maketitle

\section{Introduction}

The category of coherent sheaves is not the ideal invariant for complex varieties, because it is usually too strong. Indeed, we have the following theorem.

\begin{thm}
(Gabriel) If $\Coh(X) \simeq \Coh(Y)$, then $X \simeq Y$.
\end{thm}

Instead, given a variety $X$ over $\CC$, one might want to consider the derived category of coherent sheaves
\[
D(X) := D^b(\Coh(X)).
\]
This is slightly more flexible. A theorem of Bondal--Orlov says that the same conclusion as above holds, provided your variety is Fano or anti-Fano, but not all cases are like this.

If $X$ is smooth and projective variety over $\CC$, the category $D(X)$ has the following properties:
\begin{itemize}
  \item{it is additive}
  \item{has a shift operator $[1]$}
  \item{is equipped with the natural exact triangles}
  \item{have finite-dimensional $\oplus_i \Ext^i(E,F) = \oplus_i \Hom(E, F[i])$}
  \item{(``Serre duality") $\Hom(E,F) = \Hom(F, E \otimes \omega_X[\dim X])^*$, which inspires the definition of the \textit{Serre functor}
  \[
  S_X(E) := E \otimes \omega_X[\dim X]
  \]}
\end{itemize}

\begin{rem}
If we have a decomposition $X = Y \amalg Z$, we have
\[
i_* i^* E \to E \to j_* j^* E.
\]
\end{rem}

\begin{defn}
We say that $\A, \B \subset D(X)$ form an \textbf{orthogonal decomposition} if
\begin{itemize}  
  \item{$\Hom(\B, \A) = 0 = \Hom(\A, \B)$}
  \item{for all $E \in D(X)$, there exists a unique exact triangle
  \[
  B \to E \to A
  \]
  for all $A \in \A$ and $B \in \B$.}
\end{itemize}
\end{defn}

The definition has the following immediate consequence.

\begin{prop}
A variety $X$ is disconnected if and only if $D(X)$ admits an orthogonal decomposition.
\end{prop}

However, this does not allow for any intermediate between connected and disconnected, so we relax some hypotheses and also consider the following definition.

\begin{defn}
We say that $\A, \B \subset D(X)$ form an \textbf{semiorthogonal decomposition} if
\begin{itemize}
  \item{$\Hom(\B, \A) = 0$}
  \item{for all $E \in D(X)$, there exists a unique exact triangle
  \[
  B \to E \to A
  \]
  for all $A \in \A$ and $B \in \B$.}
\end{itemize}
We write this as $D(X) = \left<\A, \B\right>$.
\end{defn}

In general, we have
\[
D(X) = \left<\A_1, \dots, \A_k\right>.
\]
for some $\A_i \subset D(X)$.

Given an $E \in D(X)$, we define a map
\begin{align*}
\phi_E: D(\text{Vect}) = D(\pt) & \to D(X) \\
 V & \mapsto V \otimes_\CC E.
\end{align*}

\begin{defn}
We say that $E$ is a \textbf{exceptional} if $\phi_E$ is fully faithful.
\end{defn}

Generally, it's hard to find exceptional objects. However, we do have the following fundamental result.

\begin{thm}(Beilinson)
\[
D(\PP^n) = \left<\Oh, \Oh(1), \ldots, \Oh(n) \right>
\]
where $\Oh(k) = \phi_{\Oh(k)}(D(\pt))$.
\end{thm}


\begin{warn}
The standard notation is a little misleading.
We write
\[
D(\PP^1) = \left<D(\pt), D(\pt)\right>,
\]
but also
\[
D(\pt \amalg \pt) = \left< D(\pt), D(\pt) \right>.
\]
The difference is that the former is a semiorthogonal decomposition, while the latter is actually an orthogonal decomposition.
\end{warn}

\begin{rem}
If $X$ is Calabi--Yau, then $D(X)$ has no semiorthogonal decomposition.
\end{rem}

We also have a relative version of Beilinson's theorem.

\begin{thm}(Orlov)
If $E$ is a rank $n+1$ vector bundle on a variety$X$ and
\[
p: \PP_X(E) \to X,
\]
then
\[
D(\PP(E)) = \left< p^* D(X), p^*D(X) \otimes \Oh_P(1), \ldots, p^*D(X) \otimes \Oh_P(n)\right>.
\]
\end{thm}

We have a similar decomposition for blowups.

\begin{thm}(Orlov)
If $Y \subset X$ are smooth varieties and
\[
\xymatrix{
E \ar[r]^\iota \ar[d]^p & \Bl_Y X \ar[d]^\pi \\
Y \ar[r] & X
}
\]
then
\[
D(\Bl_Y X) = \left<\pi^* D(X), \iota_*p^*D(Y) \otimes \Oh_P(1), \ldots, \iota_* p^*D(Y) \otimes \Oh_P(c-2)\right>
\]
where $c = \dim_X Y$.
\end{thm}

This decomposition is sometimes useful for computing other invariants. For instance, if $D(X) = \left<\A_1, \ldots, \A_r\right>$, then $K_0(X) = \oplus_i K_0(\A_i)$.

As a final note, it is generally difficult to tell when a decomposition is full. One reason is because there are examples of ``phantom" or ``quasi-phantom" phenomena for some of these categories; they are named because they ``vanish" if we take some coarser invariants like Hochschild homology, higher K-theory, etc.


\section{Homological Projective Duality (Kuznetsov)}

The idea behind homological projective duality begins with the following question: if I understand what happens for hypersurface in a projective variety and how it moves around, what can I say about the whole derived category?

\begin{ex}
Consider $X = \{s = 0\} \subset \PP^n$ where $s$ is of degree $d \leq n$. Then
\[
D(X) = \left<\A_X, \Oh_X, \ldots, \Oh_X(n-d)\right>.
\]

The only interesting invariant in this situation is $\A_X$, so for the remainder of the talk, we will use a convenient nonstandard notation:
\[
D_{int}(X) = \A_X.
\]
\end{ex}

A natural question to ask is: What is the Serre functor $S_\A$? Nobody knows, but we do know what a \textit{power} of the Serre functor should be.

\begin{lem}
\[
S_\A^{d/\gcd(n+1,d)} = \left[ \frac{(d-2)(n+1)}{\gcd(n+1, d)}\right]
\]
\end{lem}

\begin{ex}
For cubic 4-folds, we have $n = 5$ and $d = 3$, so
\[
S_\A = [2].
\]
\end{ex}

A question one may ask in this specific situation is: when is $\A$ the derived category of a K3 surface? It is conjectured that this holds precisely for rational cubic 4-folds, guided by a geometric heuristic. However, remember that we don't have any written example of an irrational cubic 4-fold. There is also a conditional proof of this conjecture, but it assumes some strong results.


\begin{rem}
Right now you might ask, I don't see any duals anywhere, where exactly is the ``duality" in homological projective duality?

We will try and make the reason clear in what follows. The general idea is that you get some categories by varying a hypersurface is a linear family, and by considering these categories together, the base locus has ``interesting part" $\A$, so you can sort of think of $\A$ as the ``dual." 
\end{rem}

Consider $\PP^n \to \PP(V)$ where $V = H^0(\PP^n, \Oh(d))^*$. Setting
\[
\H = \{(p,s) \mid s(p) = 0\},
\]
we have
\[
\xymatrix{
\H_s \ar[r] \ar[d] & \H \subset \PP^n \times \PP(V^*) \ar[d] \\
\bullet \ar[r]& \PP(V^*)
}
\]
Then
\begin{align*}
D(\H) &= \left<\A, D(\PP(V^*) \boxtimes \Oh(d), \ldots, D(\PP(V^*)) \boxtimes \Oh(n) \right> \\
D(\H_s) &= \left<\A_s, \ldots \right>
\end{align*}

Given an $L \subset V^*$, we have
\[
\xymatrix{
\H_L \ar[r] \ar[d] &  \H \ar[d] \\
\PP(L) \ar[r] & \PP(V^*)
}
\]
with
\[
D_{int}(\H_L) = \A_L.
\]

Given $0 \to L^\perp \to V \to L^*$, we have
\[
\xymatrix{
X_L \ar[r] \ar[d] & \PP^n  \ar[d] \\
\PP(L^\perp) \ar[r] & \PP(V)
}
\]
and
\[
D_{int}(X_L) = \A_L.
\]
Note that $X_L$ is the base locus of $L$.


From this, we give the following vague definition.

\begin{defn}
A \textbf{homological projective dual} ($HP^\vee$) of $\PP^n \to \PP(V)$ is some $Y \to \PP(V^*)$ such that $D(Y) = \A$. 
\end{defn}

\begin{ex}
If $d = 1$, then $HP^\vee = \emptyset \subset \PP(V^*)$.
\end{ex}

\begin{ex}(Kapranov, reinterpreted by Kuznetsov) 
If $d = 2$, with $\{s = 0\} = Q \subset \PP^n$, then
\[
D(Q) = \left<D(Cl_0(s)), \ldots\right>
\]
where $Cl_0(s)$ is the appropriate Clifford algebra. We have $D(Cl_0(s)) \approx D(\pt \amalg \pt)$, but this identification $\approx$ is noncanonical, and the two (spinor) bundles switch as you vary in a family. Therefore,
\[
HP^\vee = (V^*, Cl_0).
\]

Given two such quadrics $Q_0$ and $Q_1$ that span some pencil $L \subset V^*$, so $Q_0 \cap Q_1$ is the base locus and
\begin{align*}
D_{int}(Q_0 \cap Q_1) & = D_{int}(\H_L) \\
&= D(\PP(L), Cl_0) \\
&= D(C)
\end{align*}
where $C$ arises from a double cover of $\PP(L) = \PP^1$.
\end{ex}

When $\dim L =3$,
\begin{align*}
D_{int}(Q_0 \cap Q_1 \cap Q_2) &= D(\PP(L), Cl_0)\\
&= D(K3, \alpha)
\end{align*}
where the final category consits of ``$\alpha$-twisted sheaves" and the K3 surface is obtained as a branched cover of $\PP(L) = \PP^2$. The Brauer class $\alpha$ is needed because of the non-canonical identification $D(Cl_0) \approx D(\pt \amalg \pt)$, where the two spinor bundles can move around and switch.

However, is no Azumaya algebra in general case (for more than four quadrics), and things can get complicated. The definitive reference for such results in Addington's thesis \textit{Spinor sheaves and complete intersections of quadrics}.



We conclude with an interesting result on the relation between orthogonal decomposition and its relation to Brauer classes.

\begin{thm}(Bernardara)
Consider a $\PP^n$-bundle
\[
p: Y \to X
\]
i.e. a bundle such that \'etale-locally on $X$, we have $p^{-1}(U) = U \times \PP^n$ and let $\beta$ denote the associated Brauer class.  Then
\[
D(Y) = \left<p^*D(X), p^*(D(X, \beta) \otimes \Oh_p(1), \ldots, p^*D(X, \beta^n) \otimes \Oh_p(n)\right>,
\]
where the relative bundles $\Oh_p(k)$ are interpreted as (appropriately) twisted sheaves.
\end{thm}



\end{document}