\documentclass[11pt]{amsart}
\usepackage{geometry}
\geometry{letterpaper}
\usepackage[parfill]{parskip}    
\usepackage{graphicx}
\usepackage{xypic}
\usepackage{amssymb}
\usepackage{epstopdf}
\usepackage{mathrsfs}
\usepackage{MnSymbol}
\usepackage{enumerate}
\DeclareGraphicsRule{.tif}{png}{.png}{`convert #1 `dirname #1`/`basename #1 .tif`.png}


\numberwithin{equation}{section}

\setcounter{tocdepth}1
\numberwithin{subsection}{section}
\renewcommand\qedsymbol{{\ensuremath{\clubsuit}}}
\renewcommand\bibname{References}
\allowdisplaybreaks[1]

%%%%%%%%%%%%%%%%%%%%%%%%%%%%%%%%%%%%%%%%%

\newenvironment{enumeratea}
{\begin{enumerate}[\upshape (a)]}
{\end{enumerate}}

\newenvironment{enumeratei}
{\begin{enumerate}[\upshape (i)]}
{\end{enumerate}}

\newenvironment{enumerate1}
{\begin{enumerate}[\upshape (1)]}
{\end{enumerate}}

\newtheorem*{namedtheorem}{\theoremname}
\newcommand{\theoremname}{testing}
\newenvironment{named}[1]{\renewcommand\theoremname{#1}
\begin{namedtheorem}}
{\end{namedtheorem}}

\newtheorem*{maintheorem}{Theorem}

\newtheorem{theorem}[subsection]{Theorem}
\newtheorem{thm}[subsection]{Theorem}
\newtheorem{proposition}[subsection]{Proposition}
\newtheorem{prop}[subsection]{Proposition}
\newtheorem{proposition-definition}[subsection]
{Proposition-Definition}
\newtheorem{corollary}[subsection]{Corollary}
\newtheorem{lemma}[subsection]{Lemma}
\theoremstyle{definition}
\newtheorem{definition}[subsection]{Definition}
\newtheorem{question}[subsection]{Question}
\newtheorem{conjecture}[subsection]{Conjecture}
\newtheorem{notation}{Notation}
\newtheorem{example}{Example}
\newtheorem{remark}{Remark}
\newtheorem{note}[subsection]{Note}
\newtheorem{problem}[subsection]{Problem}
\newtheorem*{pf}{Proof}

\theoremstyle{remark}
\newtheorem*{claim}{Claim}




\newcommand{\nc}{\newcommand}
\nc{\rnc}{\renewcommand}
\nc{\eps}{\epsilon}
\nc{\C}{\mathbb C}
\nc{\R}{\mathbb R}
\nc{\Q}{\mathbb Q}
\nc{\Z}{\mathbb Z}
\nc{\N}{\mathbb N}
\rnc{\O}{\mathcal O}
\nc{\F}{\mathcal F}
\nc{\D}{\mathcal D}
\nc{\K}{\mathcal K}
\nc{\Cp}{\mathcal C_8}
\nc{\Qu}{\mathcal Q}

\nc{\us}{\underset}
\nc{\os}{\overset}
\nc{\ul}{\underline}
\nc{\ov}{\overline}
\nc{\Ra}{\Rightarrow}
\nc{\La}{\Leftarrow}
\nc{\LRa}{\Leftrightarrow}
\nc{\bs}{\backslash}

\nc{\id}{\text{id}}
\nc{\g}{\mathfrak g}
\rnc{\L}{\mathcal L}
\nc{\tr}{\text{tr}}
\nc{\ad}{\text{ad}}
\rnc{\P}{\mathbb P}
\nc{\Bl}{\text{Bl}}
\nc{\M}{\mathcal M}
\nc{\Pic}{\text{Pic}}
\nc{\Exc}{\text{Exc}}
\nc{\Proj}{\mathscr P}
\nc{\Isom}{\mathscr I som}
\nc{\bu}{\bullet}
\nc{\A}{\mathscr A}
\nc{\Ad}{\mathbb A}

\title{Brauer-Manin Obstructions}
\author{Jennifer Park}
\date{}

\begin{document}

\maketitle

\section{Hasse Principle}

Example 1.1:  Does the variety defined by $x^2+y^2=3z^2$ in $\P^2_\Q$ have any $\Q$-points?  The answer is no, as can be seen by reducing the equation mod 9.  This implies that there are in fact no $\Q_3$-points.\\\\
Example 1.2:  Does the variety defined by $x^2+y^2+z^2=0$ in $\P^2_\Q$ have any $\Q$-points?  The answer is no again, since there are in fact no $\R$-points.  Think of $\R$ as $\Q_{\infty}$.\\\\
Now let $X$ be an arbitrary variety.  If $X(\Q_p)\neq \emptyset$ for all places $p$ implies that $X(\Q)\neq\emptyset$, we say that $X$ satisfies the Hasse Principle (HP).  Note that we always have 
$$X(\Q)\subset \prod_{p\leq\infty} X(\Q_p).$$
The following classes of varieties are known to satisfy HP:\\

$\bu$ Smooth projective quadrics (Hasse-Minkowski).\\
$\bu$ Severi-Brauer varieties (Ch\^{a}telet).\\
$\bu$ Smooth projective cubics in $\P^n_\Q$ for $n\geq 9$ (Hooley).\\
$\bu$ Smooth proper models of intersections of two quadrics in $\P^n_\Q$ for $n\geq 3$.\\

\section{Brauer-Manin}

Definition 2.1:  Let $F$ be a field.  The Brauer group of $F$ is given by the set
$$Br(F) := \{\text{f.d. central simple algebras / }F\}/\sim$$
where $A\sim B$ if $M_n(A)\simeq M_m(B)$ for some $n,m\in \N$.  Tensor product makes it into a group, with $[F]$ as the identity and inversion given by $A\mapsto A^{op}$.\\

Definition 2.2:  Let $X$ be a geometrically irreducible scheme over $k$ a field of char. 0.  The Brauer group of $X$ is given by
$$Br_{az}(X) := \{\text{Azumaya algebras on } X\}/\sim$$
where $\A\sim \mathscr B$ if $\A\otimes \mathcal E\mathcal n\mathcal d(\mathcal E) \simeq \mathscr B \otimes \mathcal E\mathcal n \mathcal d(\mathcal F)$ for locally free sheaves $\mathcal E, \mathcal F$ on $X$ of finite positive rank.  There is also a cohomological definition
$$Br_{coh}(X) := H^2_{\text{\'{e}t}}(X,\mathbb G_m)$$
and the two notions agree when $X$ is smooth projective.\\

Example 2.3:  Let $a,b\in \Q^\times$.  We define the quaternion algebra $(a,b)$ as
$$\Q\{i,j\}/(i^2=a, j^2=b, ij=-ji)$$
One can check that the Brauer class of $(a,b)$ lies in $Br(\Q)[2]$.  Similarly, let $(a,b)_p$ denote the algebra $(a,b)\otimes \Q_p$, whose Brauer class lies in $Br(\Q_p)[2]$.  

Facts:\\
$\bu$ $(ab,c)=(a,c)(b,c)$\\
$\bu$ $(a,b)=(b,a)$\\
$\bu$ $(a,1-a)=1$\\

Now let $\A\in Br_{az}(X)$, and let $\Ad$ denote the rational adeles.  The following diagram commutes:
$$\xymatrix{
{} & X(\Q) \ar[d]_{\A}\ar[r] &X(\Ad) \ar[d]_{\A} & {}& {} \\
0 \ar[r] & Br(\Q)\ar[r] & \bigoplus_{p\leq\infty} Br(\Q_p) \ar[r] & \Q/\Z \ar[r] & 0
}$$
Here, the vertical arrows take the fiber of the Azumaya algebra $\A$ over each point, and the right hand map is given by local class field theory
$$\sum_p inv_p :  \bigoplus_{p\leq\infty} Br(\Q_p) \to  \Q/\Z.$$
Global class field theory says that the bottom sequence is exact.  Now, set
$$X(\Ad)^\A := \{(x_p)\in X(\Ad): \sum_p inv_p(\A(x_p))=0\}$$
$$X(\Ad)^{Br} := \bigcap_{\A\in Br(X)} X(\Ad)^\A$$
and observe that
$$X(\Q)\subset X(\Ad)^{Br}\subset X(\Ad)^\A\subset X(\Ad)$$
If $X(\Ad)^{Br}\neq\emptyset$ implies that $X(\Q)\neq\emptyset$, we say that Brauer-Manin (BM) is the {\it only} obstruction to the Hasse Principle.\\

Example 2.4:  Let $X$ be the del Pezzo surface in $\P^4_\Q$ defined by 
$$uv = x^2-5y^2$$
$$(u+v)(u+2v)=x^2-5z^2.$$
This does not satisfy the HP, but there is a Brauer-Manin obstruction.  To see this, first recall that $(a,b)=1$ if $a$ appears as a norm in $\Q(\sqrt{b})$.  In fact, $X(\Ad)^\A=\emptyset$ for 
$$\A=\left(5,\frac{u}{u+v}\right) = \left(5,\frac{u}{u+2v}\right) = \left(5,\frac{v}{u+v}\right) =\left (5,\frac{v}{u+2v}\right)\in Br(X).$$
$\bu$ $inv_{\infty} \A_\infty=0$, since 5 is a square in $\R$.\\
$\bu$ Similarly, if $p$ is an odd prime with $(\frac{5}{p})=1$, $inv_p \A_p=0$.\\
$\bu$ If $p$ is an odd prime with $(\frac{5}{p})=-1$, then $uv \equiv (u+v)(u+2v)$ mod 5, so one of the four expressions for $\A_p$ is trivial, so $inv_p\A_p=0$.\\
$\bu$ If $p=2$, look at all possibilities mod 16 to find $inv_2 \A_2=0$.\\
$\bu$ If $p=5$, we have $u\equiv v\equiv \pm x$ mod 5, so
$$\frac{u}{u+v}\equiv 3\text{ mod 5}.$$
Hence, $(5,3)_5=-1$ so $inv_5 \A_5=1/2$.\\

Putting this all together, we find that
$$\sum_p inv_p(\A_p) = 1/2$$
so $X(\Ad)^\A=\emptyset$.  $\blacksquare$.\\

For the following classes of varieties, BM is the only obstruction to the HP:\\

$\bu$ Smooth proper models of cubic hypersurfaces with 2 or 3 singular points, over $\Q$.\\
$\bu$ Singular cubic threefolds with isolated singularities.\\
$\bu$ Ch\^{a}telet surfaces.\\

Conjecture 2.5:  Curves and rationally conn. sm. proj. varieties are also in this list.

\section{Insufficiency of BM Obstruction}
Since BM is computable, but $\Q$-points are probably not (Hilbert's 10th Problem), the BM obstruction must be insufficient.  The first counterexample was found by Skorobogatov (1999).  Here, we sketch the idea of a simpler example discovered by Poonen (2008).\\

Let $\mathscr V\to \P^1\times \P^1$ be a conic bundle, and let $\pi:\mathscr V\to \P^1$ denote the bundle map composed with the first projection.  The fibers of $\pi$ are Ch\^{a}telet surfaces, and the fiber over $\infty$ has no $\Q$-points, but it does have adelic points:
$$\mathscr V_{\infty}(\Ad)\neq\emptyset.$$
Let $C\to \P^1$ be a hyperelliptic curve of genus $\geq 2$, so it has finitely many $\Q$-points.  Now define $X$ to be the fiber product
$$\xymatrix{
X \ar[r]\ar[d]_\beta & \mathscr V \ar[d]_\pi\\
C \ar[r]^{2:1} & \P^1
}$$
By construction, $X(\Q)=\emptyset$.  It turns out that $Br(X)\simeq Br(C)$, so then
$$X(\Ad)^{Br} = \beta^{-1}(C(\Ad)^{Br})\supset \beta^{-1}(C(\Q)) = \mathscr V_\infty(\Ad)\times C(\Q)\neq \emptyset$$

\section{\'{E}tale Brauer Obstruction}

In this section, $G\to X$ will denote a finite \'{e}tale group scheme, and $f:Z\to X$ a $G$-torsor.  We have
$$X(\Q) = \coprod_{\tau\in H^1(\Q,G)} f^\tau(Z^\tau(\Q)) \subset \bigcup_{\tau\in H^1(\Q,G)} f^\tau(Z^\tau(\Ad))$$

Definition 4.1:  The \'{e}tale Brauer adelic points of $X$ are given by
$$X(\Ad)^{\text{\'{e}t},Br} := \bigcap_{G\to X} \left(  \bigcup_{\tau\in H^1(\Q,G)} f^\tau(Z^\tau(\Ad)^{Br}) \right)$$
It is clear that
$$X(\Q)\subset X(\Ad)^{\text{\'{e}t},Br} \subset X(\Ad)^{Br} \subset X(\Ad)$$
so we get a finer obstruction.\\

Poonen (2008):  The \'{e}tale Brauer obstruction is still not sufficient.


















\end{document}