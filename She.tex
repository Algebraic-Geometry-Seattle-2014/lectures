\documentclass{amsart}

\usepackage[utf8]{inputenc}
\usepackage[T1]{fontenc}
\usepackage{eulervm}
\usepackage{tgpagella}

\usepackage{amsmath}
\usepackage{amssymb}
\usepackage{amsthm}
\usepackage{amsfonts}
\usepackage{mathrsfs}
\usepackage{xspace}
\usepackage{tikz}
\usepackage{nicefrac}
\usepackage{fixmath}
\usepackage{paralist}
\usepackage{ellipsis}
\usepackage{mathtools}
\usepackage{graphicx,subfigure,epic,eepic}
\usepackage{array}
\usepackage{tikz-cd}

\usepackage{fixltx2e}
\usepackage[expansion=false,final]{microtype}

%has to come second to last!
\usepackage[colorlinks=false, pdfborder={0 0 0}]{hyperref} 
%has to come last!
\usepackage{cleveref} 

\newcommand{\todo}[1]{ {\color{red} TODO #1}}
\renewcommand{\prime}{\mathfrak{p}}
\newcommand{\from}{\colon}

\theoremstyle{plain}
\newtheorem{theorem}{Theorem}[section]
\newtheorem{lemma}[theorem]{Lemma}
\newtheorem{proposition}[theorem]{Proposition}
\newtheorem{corollary}[theorem]{Corollary}
\newtheorem*{conjecture}{Conjecture}

\theoremstyle{definition}
\newtheorem{definition}[theorem]{Definition}

\theoremstyle{remark}
\newtheorem{example}[theorem]{Example}
\newtheorem{remark}[theorem]{Remark}

\DeclareMathOperator{\characteristic}{char}
\DeclareMathOperator{\Spec}{Spec}
\DeclareMathOperator{\Frob}{Frob}
\DeclareMathOperator{\Pic}{Pic}
\DeclareMathOperator{\SL}{SL}
\DeclareMathOperator{\Trans}{Trans}

\DeclareFontFamily{U}{wncy}{}
\DeclareFontShape{U}{wncy}{m}{n}{<->wncyr10}{}
\DeclareSymbolFont{mcy}{U}{wncy}{m}{n}
\DeclareMathSymbol{\Sh}{\mathord}{mcy}{"58} 
\DeclareMathOperator{\Sha}{ \Sh }

%\DeclareMathOperator{\Sha}{ {\fontencoding{OT2}\selectfont SH} }

\title{Arithmetic of K3 surfaces}
\author{Yiwei She, notes by Gabriel Bujokas}
\date{August 8, 2014}

\begin{document}

\maketitle

We start with the following motivating conjecture.
\begin{conjecture}
There are infinitely many rational
curves on a K3 surface defined over an algebraically closed field.
\end{conjecture}

This fits in the broader context of the Lang's conjecture, which says that all rational curves in a variety of general type should be contained in a proper subvariety. Both surfaces and Calabi-Yau varieties are good places to figure out what the analogue of Lang's conjecture in the non-general type varieties should say. 
The class of K3 surfaces is the smallest case in which we don’t know the answer.

\section{Basic Properties of K3 surfaces}
Let $X$ be an algebraic K3 surface, with polarization $L$. Then:
\begin{itemize}
	\item By definition, the canonical divisor $K_X$ is trivial, and $h^{1,0}(X)=0$.
	\item The Hodge diamond is the following:
	\[
	\begin{matrix}
	  &   & 1  &   & \\
	  & 0 &    & 0 & \\
	1 &   & 20 &   & 1 \\
	  & 0 &    & 0 & \\
	  &   &  1 &   &
	\end{matrix}
	\]
	\item As a lattice with intersection pairing, $H^2(X,\mathbb{Z})$ is isomorphic to 
	$\mathbb{H}^3 \oplus (E_8)^2$.
	%\item The group $\Pic^0(X)=\set{D \in \Pic X \text{ such that } DL=0}$ is trivial (Rizzo).
	% Hm, the \Pic^0 could mean the component of $L$ of $\Pic X$...
	\item The Picard group $\Pic(X)$ is a sublattice of 
	$H^2(X , \mathbb{Z}) \cap H^{1,1}(X ,\mathbb{C})$. Generically, $\Pic (X)= \mathbb{Z}$
	generated by the polarizing class $L$. But any sublattice of signature $(1,-k)$ can be realized,
	as long as $k \leq 19$.
	\item The transcendental lattice $\Trans(X)= \left(\Pic(X) \right)^\perp  = \mathbb{Z}^{22-\rho}$, where
	$\rho$ denotes the rank of the Picard group $\Pic(X)$.
\end{itemize}

\section{Rational curves on K3 surfaces} % (fold)
\label{sec:rational_curves_on_k3_surfaces}

The first existence results are the following.
\begin{theorem}[Bogomolov--Mumford 70's]
\label{theorem:bogomolov-mumford}
Let $X$ be a polarized K3 surface over an algebraically closed field.
Then $X$ contains at least one rational curve.
 More specifically, any effective divisor class $D$ has a representative supported on a union of
 rational curves. 
\end{theorem}
\begin{remark}
We do not get, a priori, infinitely many rational curves by varying the class $D$, 
because the representative of $D$ produced by the theorem may not be reduced.
\end{remark}

\begin{theorem}[Bogomolov--Tschinkel 99']
\label{theorem:bogomolov-tschinkel}
An elliptic K3 surface over an algebraically closed field contains infinitely many rational curves.
\end{theorem}

A K3 surface $X$ is elliptic if there is a map $X \to \mathbb{P}^1$ such that the generic
fiber is a smooth curve of genus one. Alternatively, one may characterize elliptic K3 surfaces as the ones that admit a non trivial line bundle $L$ of self-intersection $0$. Note that this is purely a condition on
the lattice $\Pic(X)$. As a matter of fact, one can show that if the Picard rank $\rho$ is at least five, then one may always find a self-intersection zero divisor, and hence the K3 surface is elliptic.

A quick Euler characteristic computation shows that not all fibers of $X \to \mathbb{P}^1$ are smooth.
As a matter of fact, one can prove much more: Kodaira showed that an elliptic K3 surface has at least
four singular fibers, and at least one of potentially multiplicative reduction.

We call an elliptic K3 surface a \emph{Jacobian} if
the fibration admits a section $s_0:\mathbb{P}^1 \to X$. Given an arbitrary elliptic
K3 surface $E \to \mathbb{P}^1$, one may produce a Jacobian fibration $J(E) \to \mathbb{P}^1$,
where for generic $t \in \mathbb{P}^1$,
$J(E)_t = \Pic^0(E_t)$. Note that $J(E)$ does have a section, sending $t$ to $\mathcal{O}_{E_t} \in \Pic^0(E_t)$.


For the purpose of this talk, we define the Tate--Shafarevich group $\Sha(X/\mathbb{P}^1)$ of 
a Jacobian fibration $X/\mathbb{P}^1$ as the set of isomorphism classes of elliptic fibrations
$E$ such that $J(E)$ is isomorphic to $X$.

\begin{theorem}[Shafarevich, 60's]
$\Sha(X/\mathbb{P}^1)=\Trans(X) \otimes \mathbb{Q}/\mathbb{Z}$
\end{theorem} 


\begin{proof}[Sketch of proof of \Cref{theorem:bogomolov-tschinkel}]
For simplicity, let us go over the proof in the case $X/\mathbb{P}^1$ is Jacobian. Take
an element $0\neq [\frac{1}{n}] \in \Sha(X/\mathbb{P}^1)$, and let $E_{\frac{1}{n}}$ be the corresponding 
elliptic K3.
\begin{lemma}
There is a dominant map  $E_{\frac{1}{n}} \to \Pic^n(E_{1/n}) \cong X$.
\end{lemma}
\begin{remark}
The isomorphism $\Pic^n(E_{1/n}) \cong X$ is not canonical.
\end{remark}
In fact, the map is essentially multiplication by $n$ fiberwise.

By taking a sequence $[\frac{1}{n^k}]$, for $k=1,2,\ldots$, we get an infinite chain
\[
	\ldots  \to E_{\frac{1}{n^2}} \to E_{\frac{1}{n}} \to X
\]
Now \Cref{theorem:bogomolov-mumford} produces a rational curve in each of the $E_{\frac{1}{n^k}}$, and 
its image in $X$ will be supported on a rational curve as well. One then argues that it is impossible for infinitely many of these images to be supported on the same rational curve in $X$.
\end{proof}

With a little bit more work, we will get the following.
\begin{corollary}
\label{corollary:potential-density}
An elliptic K3 surface $X$ defined over a number field $k$ is potentially dense. That is,
there exist a finite extension $K/k$ such that the $K$-points of $X$ are Zariski dense.
\end{corollary}
Potential density of varieties is an interesting and well-studied question. In the case of curves, 
it has been completely answered:
\begin{itemize}
	\item Genus zero curves are potentially dense (and one might argue that this goes back to
	Pythagoras),
	\item Genus one curves are also potentially dense, by a theorem of Frey,
	\item Genus two or higher are \emph{not} potentially dense, by Falting's theorem. 
\end{itemize}

We know less in the case of surfaces. Of course, rational surfaces are still potentially dense, and so
are abelian surfaces by Frey's theorem again. 

\begin{proof}[Proof of \Cref{corollary:potential-density}]
Let us assume again that $X$ is Jacobian. The idea of the proof is to produce a second section 
$s_{\infty}: \mathbb{P}^1 \to X$, and use the group structure of the generic fiber to produce countably many ones. As a matter of fact, we can even get by with a rational multi-section $C$, as long as not all points 
$C_t \in X_t$ are torsion. This is ruled out by the following proposition.
\begin{proposition}
 \label{proposition:torsion}
 There exists an $m_0$ such that any torsion multi-section $C_m$ of order $m>m_0$ has genus at least $2$.
 \end{proposition}
 On the other hand, there are finitely many torsion multi-sections of order at most $m_0$, but
 \Cref{theorem:bogomolov-tschinkel} already produced infinitely many rational curves.
\end{proof}

\begin{proof}[Proof of \Cref{proposition:torsion}]
Let $\Gamma \subset \SL_2(\mathbb{Z})$ be the monodromy group of the elliptic surface $X/\mathbb{P}^1$.
Let $\Gamma_m \subset \Gamma$ be the subgroup which fixes a point of $m$-torsion.
Generically, $C_m \to \mathbb{P}^1$ has degree $[\Gamma_m:\Gamma]$. Hence,
\[
	\chi(C_m)=[\Gamma_m:\Gamma]\left( 2- \sum_{\text{ramification}} (1- \frac{1}{\text{ramification index}})
	\right)
\]
The ramification is supported at the singular fibers. If the singular fiber is
\begin{itemize}
	\item $\mathbb{G}_m$, there are only $m$ $m$-torsion points, instead of the usual $m^2$. Hence,
	the ramification index is $m$.
	\item Otherwise, the local monodromy group has order $2,3,4$ or $6$. In any case, 
	\[
		1- \frac{1}{\text{ramification index}} = 1 - \frac{1}{\text{$2,3,4$ or $6$}} \geq \frac{1}{2}
	\]
\end{itemize}

As Kodaira showed that we get at least one of the first, and at least 4 singular fibers, we get
\[
	\sum_{\text{ramification}} (1- \frac{1}{\text{ramification index}}) \geq 1-\frac{1}{m} +3 \times \frac{1}{2}> 2
\]
for $m \geq 3$. Therefore, $\chi(C_m)<0$, as we wanted to show.
\end{proof}

% section rational_curves_on_k3_surfaces (end)

\section{Rational curves on the odd Picard rank case} % (fold)
\label{sec:rational_curves_on_the_odd_picard_rank_case}

The generic K3 surface has Picard rank $1$, and \Cref{theorem:bogomolov-tschinkel} and 
\Cref{corollary:potential-density} do not apply. It is still not known whether 
generic K3 surfaces are potentially dense.

However, we do know that there are infinitely many rational curves in $\bar{\mathbb{Q}}$,
by recent work of Bogomolov--Hasset--Tschinkel and Li--Liedtke in 2013.
\begin{theorem}[BHT, LL]
Let $X$ be an odd Picard rank K3 surface over an algebraically closed field of characteristic zero.
Then there are infinitely many rational curves on $X$.
\end{theorem}
\begin{remark}
Li--Liedtke also extended the proof to algebraically closed fields of characteristic at least $5$, as
long as the K3 surface $X$ is not supersingular.
\end{remark}
\begin{proof}
A standard argument reduces to the case where $X$ is defined over a number field $k$. Let's us assume this is the case.

Next, we will use a mixed characteristic argument: we will show that
there are rational curves in the prime reductions, and that one can use deformation theory to lift them
to the characteristic zero case.

The advantage of the finite characteristic case is the following consequence of Tate's conjecture.
\begin{lemma}
\label{lemma:tate}
If $X$ is a K3 surface defined over a finite field $\mathbb{F}_q$,
 then the geometric Picard rank of $X$ is even.
\end{lemma}
\begin{proof}
Let us assume $\characteristic \mathbb{F}_q >3$, and that $X$ is ordinary. Then
\[
	\Pic X_{\overline{\mathbb{F}_q}} = H^2_{\text{et}}(X)^{\Frob_q}
\]
The eigenvalues of Frobenius come in pairs $\alpha, \alpha^{-1}$. Hence the total multiplicity of the
eigenvalues $\pm 1$ 
is $22-2k$, where $k$ is the number of pairs. In particular, $\rho$ is even.
\end{proof}

Choose an integral model $\mathcal{X} \to \Spec(\mathfrak{o}_k)$ of our K3 surface over the number
field $k$. Denote by $h$ the polarization of $\mathcal{X}$. 
As the Picard rank is upper-semicontinuous, \Cref{lemma:tate} implies that the rank jumps up
whenever we restrict to a prime $\prime$ of $\mathfrak{o}_k$. 
Let $C_{\prime}$ be a primitive effective class on $\mathcal{X}_\prime$ not coming from
$\Pic(X_{\bar{k}})$.

\begin{lemma}
We may choose the classes $C_{\prime}$ such that the degree $h\cdot C_{\prime}$ is arbitrarily high.
\end{lemma}
\begin{proof}
For each fixed degree $d=h \cdot C_{\prime}$, the Hilbert scheme of curves in fibers of
$\mathcal{X} \to \Spec(\mathfrak{o}_k)$ of degree $d$ is proper. 
The components that dominate $\Spec(\mathfrak{o}_k)$
must have linear classes in $\Pic(X_{\bar{k}})$. There can be components of the Hilbert scheme contained
in fibers of the map to $\Spec(\mathfrak{o}_k)$, but by properness only finitely many of those.
Picking any other prime will provide classes with $h\cdot C_{\prime}>d$.
\end{proof}

Now the plan is: for each prime $\prime$, choose $n$ large enough so that $nh-C_{\prime}$ is effective.
Then by \Cref{theorem:bogomolov-mumford}, we may choose representatives of $C_{\prime}$ and $nh-C_{\prime}$
supported on rational curves. That is, we may write $nh$ as a sum of rational curves $C_{\prime}+\sum C_i$.
If we can deform this linear combination to a tree of rational curves in the generic fiber, then one of its
irreducible components will have degree larger than $h \cdot C_{\prime}$, which can be arbitrarily high. This implies that there are infinitely many rational curves in $X_{\bar{k}}$, as we wanted to show.

We study the deformations of $D=C_{\prime}+\sum C_i$ by constructing a stable map $f: R \to \mathcal{X}$ from
a tree of rational curves $R$,
whose image is $f(R)=D$.
Bogomolov--Hasset--Tschinkel can compute the dimension of the moduli space of stable maps
at the point $[f]$. If we can show that there are no non-trivial 
deformations of $[f]$ contained in the fiber $\mathcal{X}_{\prime}$, then by dimensional reasons there must
be a deformation of $[f]$ into the generic fiber of $\mathcal{X} \to \Spec(\mathfrak{o}_k)$, as we wanted to show. 
That is, our goal is to choose a map $f$ that is \emph{rigid} in the central fiber $\mathcal{X}_{\prime}$.

The difficulty is that we have no control over the geometry of $D$---it may be non-reduced, with arbitrarily bad singularities. Fortunately, there is an easy criterion for rigidity of rational curves in K3 surfaces.
\begin{proposition}
\label{proposition:rigid-conditions}
A stable map $f:R \to \mathcal{X}_{\prime}$ is rigid if
\begin{itemize}
	\item $R$ is a nodal arithmetic genus zero curve,
	\item for each component $R_i$ of $R$, the map $f|_{R_i}$ is birational onto its image, and
	\item for each node $\eta \in R$, the image under $f$ of the 
	two components of $R$ containing $\eta$ meet properly at $f(\eta)$.
\end{itemize}
\end{proposition}
We can't find a rigid map over an arbitrary $D$, but we can construct one for $D+kR$, where $R$ is a 
\emph{rigidifier}.
\begin{definition}
A rigidifier is a morphism $f\from \mathbb{P}^1 \to X$ to a surface such that
\begin{itemize}
	\item $f\from \mathbb{P}^1 \to f(\mathbb{P}^1)=R$ is birational,
	\item $R$ has at most nodal singularities,
	\item and the class of $R$ in $X$ is ample.
 \end{itemize}
\end{definition}
The existence of rigidifiers (for generic $\prime$) ultimately follows from Chen's proof of the existence of nodal
rational curves in generic K3 surfaces. Now we are left with showing the following.
\begin{proposition}
For any effective divisor $D \subset \mathcal{X}_\prime$ which is supported on integral rational curves, 
there is a rigid map $f:C \to \mathcal{X}_\prime$ from a nodal connected arithmetic genus zero curve $C$,
such that $f(C)=D+kR$, where $R$ is a rigidifier.
\end{proposition}
We give a ``proof by picture'' of this proposition (but unfortunately the picture will be suppressed from the notes). The idea is to take $f_D:\tilde{D} \to \mathcal{X}_\prime$ a disconnected union of the normalization 
of the components of $D$ (taking the appropriate number of copies of each normalization, so that $f_D(\tilde{D})=D$). Now we glue several copies of the rigidifier $\mathbb{P}^1 \to R \subset \mathcal{X}_\prime$ to
make $\tilde{D}$ a connected curve, but still preserving the conditions of \Cref{proposition:rigid-conditions}.
For example, if $f:\mathbb{P}^1 \to R$ is a rigidifier and $f(t_1)=f(t_2)=\eta$ is a node of $R$,
then we may glue two copies of the rigidifier by identifying the $t_1$ point of one with the $t_2$ point of the other. 
\end{proof}
% section rational_curves_on_the_odd_picard_rank_case (end)

\end{document}