\documentclass[reqno]{amsart}
\pagestyle{plain}
\usepackage{amsmath}
\usepackage{amscd}
\usepackage{graphics}
\usepackage{latexsym}
%\addtolength{\textwidth}{25mm}
%\addtolength{\textheight}{20mm}
%\addtolength{\oddsidemargin}{-.5\oddsidemargin}
%\addtolength{\evensidemargin}{-.5\evensidemargin}
%\addtolength{\topmargin}{-.5\topmargin}
%\addtolength{\headsep}{-.55\headsep}
%\addtolength{\footskip}{-.5\footskip}

\theoremstyle{plain}
\newtheorem{theorem}{Theorem}[section]
\newtheorem{theo}[theorem]{Theorem}
\newtheorem{cor}[theorem]{Corollary}
\newtheorem{lem}[theorem]{Lemma}
\newtheorem{prop}[theorem]{Proposition}
\newtheorem{claim}[theorem]{Claim}

\theoremstyle{definition}
\newtheorem{definition}[theorem]{Definition}
\newtheorem{notat}[theorem]{Notation}
\newtheorem{conv}[theorem]{Convention}
\newtheorem{problem}[theorem]{Problem}
\newtheorem{conj}[theorem]{Conjecture}
\newtheorem{ques}[theorem]{Question}
\newtheorem{rmk}[theorem]{Remark}

\newtheorem{ex}{Example}
\newtheorem{corex}[ex]{Corollary and Example}

\theoremstyle{remark}
\newtheorem{Remark}{Remark}[section]

\newcommand{\QQ}{\mathbb{Q}}
\newcommand{\NN}{\mathbb{N}}
\newcommand{\ZZ}{\mathbb{Z}}
\newcommand{\CC}{\mathbb{C}}
\newcommand{\RR}{\mathbb{R}}
\newcommand{\AAA}{\mathbb{A}}
\newcommand{\PP}{\mathbb{P}}
\newcommand{\PPP}{\PP_{\CC}}
\newcommand{\SP}{\text{Spec }}
\newcommand{\HB}[3]{\text{Hilb}^{#1}_{{#2}/{#3}}}
\newcommand{\GR}[3]{\text{G}_{#1}(#2,#3)}
\newcommand{\mgn}[2]{\text{M}_{#1,#2}}
\newcommand{\Mgn}[2]{\mathcal{M}_{#1,#2}}
\newcommand{\mgnb}[2]{\overline{\text{M}}_{#1,#2}}
\newcommand{\Mgnb}[2]{\overline{\mathcal{M}}_{#1,#2}}
\newcommand{\kgn}[4]{\text{M}_{#1,#2}(#3,#4)}
\newcommand{\Kgn}[4]{\mathcal{M}_{#1,#2}(#3,#4)}
\newcommand{\kgnb}[4]{\overline{\text{M}}_{#1,#2}(#3,#4)}
\newcommand{\Kgnb}[4]{\overline{\mathcal{M}}_{#1,#2}(#3,#4)}
\newcommand{\kgnbo}[4]{\overline{\text{M}}_{#1,#2}(#3,#4)^o}
\newcommand{\Kgnbo}[4]{\overline{\mathcal{M}}_{#1,#2}(#3,#4)^o}
\newcommand{\Kbm}[2]{\overline{\mathcal{M}}_{#1}(#2)}
\newcommand{\kbm}[2]{\overline{M}_{#1}(#2)}
\newcommand{\kbmo}[2]{{M}_{#1}(#2)}
\newcommand{\Kbmo}[2]{{\mathcal{M}}_{#1}(#2)}
\newcommand{\kbmoo}[2]{{M}_{#1}(#2)^o}
\newcommand{\Kbmoo}[2]{{\mathcal{M}}_{#1}(#2)^o}
\newcommand{\Na}[2]{{\mathcal{N}}(#1,#2)}
\newcommand{\nNa}[2]{{\mathcal{N}}(#1,#2)_s}


\newcommand{\lt}{\left}
\newcommand{\rt}{\right}
\newcommand{\bc}[2]{\binom{#1}{#2}}

\newcommand{\mc}{\mathcal}
\newcommand{\cB}{\mathcal B}
\newcommand{\cC}{\mathcal C}
\newcommand{\cE}{{\mathcal E}}
\newcommand{\cF}{{\mathcal F}}
\newcommand{\cG}{{\mathcal G}}
\newcommand{\cX}{{\mathcal X}}
\newcommand{\cD}{{\mathcal D}}
\newcommand{\lci}{\mathcal {LCI}}
\newcommand{\fe}{\mathcal {FE}}

\newcommand{\g}{{\frak g}}
\newcommand{\gs}{{\frak g}^{\ast}}
\newcommand{\HH}{{\cal H}}
\newcommand{\OO}{\mathcal O}
\newcommand{\Ci}{C^{\infty}}
\newcommand{\B}{{\cal B}}
\newcommand{\Cs}{C^{\ast}}
\newcommand{\Ws}{W^{\ast}}
\newcommand{\Metc}{M^{d,w}}
\newcommand{\Hetc}{H^{d,w}}
\newcommand{\Betc}{(B,\Sigma,b_0)}

\def\mapright#1{-{#1}\to}
\def\Ker{{\rm Ker}\,}
\def\dd{{\rm d}}
\def\Proj{{\rm Proj}\,}
\def\Spec{{\rm Spec}\,}
\def\Hom{{\rm Hom}\,}
\def\mC{{\mathbb C}}
\def\mP{{\mathbb P}}
\def\PP{{\mathbb P}}
\def\cO{{\mathcal O}}
\def\cZ{{\mathcal Z}}
\def\cM{{\mathcal M}}
\def\cC{{\mathcal C}}

\begin{document}

\title{Notes on Moduli spaces in Derived categories} 

\author[Bertram]{Aaron Bertram}\footnote{Lecture Notes typed by C\'esar Lozano Huerta at the conference New Connections for recent PhDs. Seattle, WA.}
\address{Department of Mathematics \\
  University of Utah \\ Salt Lake City, Utah}
\email{bertram@math.utah.edu} 
\date{\today}



\maketitle


\section*{Motivation} 


\begin{problem}
Find a moduli space for the elements of $Coh(X)$, where $X$ is a smooth, projective algebraic variety over the field of complex numbers.
\end{problem}


\subsection*{History} 

In the begining, we have discrete invariants: $Ch:K_0(X)\rightarrow V=H^{p,p}(X)\cap H^*(X,\mathbb{Q})$

\begin{itemize}
\item[(a)] Let $X=C$ be a curve. Then, $Ch(E)=(r,d)\in H^0(C)\oplus H^2(C)$.

\item[(b)] Let $X=S$ be a surface. Then, $Ch(E)=(r,c_1,ch_2)\in H^0(S)\oplus NS(S)_{\mathbb{Q}}\oplus H^4(S)=\tilde {NS}(S)$.
\end{itemize}

If we fix a Chern character $$c:=Ch(\mathcal{F}),$$ then the moduli space of coherent sheaves with fixed Chern character $\mathcal{M}_X(c)$, is a stack.
Observe that $\mathbb{C}\cdot id\subset End(\mathcal{F})$ multiples of the identity are contained in the automorphisms of $\mathcal{F}$. 

During a conference celebrating Hartshorne, it came as a surprise to me that they were analyzing the following,
\begin{definition}
Let $\mathcal{F}$ a coherent sheave. We say that $\mathcal{F}$ is simple if $End{\mathcal{F}}=\mathbb{C}\cdot id$.
\end{definition}

Let us list some examples of the behavior of the moduli space $\mathcal{M}_X^{\tiny{simple}}(c)$.
\begin{itemize}
\item Let $X=\mathbb{P}^1$. Then, $\mathcal{O}(d)$, $\mathbb{C}_x$ are simple. Moreover, $\mathcal{M}_X^{\tiny{simple}}(c)=\mathbb{P}^1$.

\item Let $E$ be an elliptic curve. Then, $\mathbb{C}_x$, line bundles are simple. In addition to these, there are simple vector bundles $E$ of coprime $(r,d)$. In this case $\mathcal{M}_X^{\tiny{simple}}(c)=E$.
\end{itemize}

In higher genus (and higher dimensions) the moduli space of simple sheaves are not separated (in general). Here is a construction which explains this.

Let $E_{\Delta}\rightarrow X$ be a family of vector bundles over a disk $\Delta$. The central fiber $E_0$ fits into the following exact sequence,
$$0\longrightarrow K_0\longrightarrow E_0\longrightarrow Q_0\longrightarrow 0 $$
Now we can construct another vector bundle via the following,
$$0\longrightarrow \tilde E_{\Delta} \longrightarrow E_{\Delta}\longrightarrow j_*Q_0\longrightarrow 0, $$ where $j:E_0\rightarrow E_{\Delta}$ is the inclusion of the central fiber.
Hence, we have another vector bundle $\tilde E_0$ that fits as a central fiber of $E_{\Delta}$,
$$0\longrightarrow Q_0\longrightarrow \tilde E_0\longrightarrow K_0\longrightarrow 0. $$
We need a tool in order to rule out vector bundles, such as $\tilde E_0$, and make the moduli space separated. 

\begin{definition}
A vector bundle $E$ on a curve $C$ is stable if its slope $\mu(E):=\frac{\mathrm{deg}E}{rk E}$ satisfies that $$\mu(F)<\mu(E) $$ for all $F\subset E$.
\end{definition}

Observe that $\mu(F)<\mu(E) $ if and only if $\mu(F)<\mu(E/F) $.

\begin{theo}($\sim$60's)
The moduli spaces of vector bundles, $\mathcal{M}_C^{\tiny{stable}}(r,d)$, of a fixed rank $r$ and fixed degree $d$ on a curve $C$, are quasi-projective.
\end{theo}

\textbf{Exercise:} If the vector bundle $E$ is stable, then $E$ is simple.

\medskip
\textbf{Exercise:} If $E$ and $F$ are stable and $\mu(E)<\mu(F)$, then $Hom(F,E)=0$.

We say that $E$ is semi-stable if there is a filtration (of slope $\mu$) $$0=E_0\subset E_1\subset \ldots \subset E_n=E,$$ such that $E_{i+1}/E_i$ are stable of the same slope.

We have that $$\mathcal{M}_C^{\tiny{stable}}(r,d)\subset \mathcal{M}_C^{\tiny{ss}}(r,d)$$. Furthermore, $E\sim E'$ if and only if $\oplus E_{i+1}/E_i\cong \oplus E'_{i+1}/E'_i$.

\section*{Abstract definition of stability}
Let $\mathcal{A}$ be an abelian category. For example $Coh(X)$.
A stability condition on $\mathcal{A}$ is a pair of functions
$$d:K(\mathcal{A})\rightarrow \mathbb{R}, \qquad r:K(\mathcal{A})\rightarrow \mathbb{R}$$
such that $r(\mathcal{A})\ge 0$ for all $\mathcal{A}\in \mathrm{Ob}(\mathcal{A})$, and $r(\mathcal{A})=0$ if and only if $d> 0$.

\begin{definition}(Rudukov) $\mathcal{A}$ is prestable if  
\begin{itemize}
\item [(1)] $\mu(B)<\mu(A) $ for all $A\subset B$, and the function$$Z:=-d+ir$$ has  the range 

$$\mathrm{P I C T U R E}$$

\item[(2)] for all $\mathcal{A}\in \mathrm{Ob}(\mathcal{A})$, there is a finite filtration (Harder-Narashiman filtration) such that $\mu(A_{i+1}/A_i)<\mu(A_1/A_{i-1})$.
\end{itemize}
\end{definition}
 

Observe that there is no stability conditions on $Coh(X)$, where dim$X>1$ (See Huybrechts' notes stability conditions).

\begin{definition}
Let $\mathcal{D}$ be a triangulated category, (eg. $\mathcal{D}^b(Coh(X))$). A stability condition on $\mathcal{D}$ is 
\begin{itemize}
\item a $t$-structure on $\mathcal{D}$,

\item a stabilitty condition on the heart of the $t$-structure.
\end{itemize}
\end{definition}

\begin{theo}(Bridgeland) Let $\mathcal{D}=\mathcal{D}^b(Coh(X))$. The map $$\mbox{stability conditions}=\mathrm{Stab}(X)\overset{(\sigma,Z)\mapsto Z}{\longrightarrow}Hom(\oplus H^{p,p}(X),\mathbb{C})$$ is \'etale on a subspace of $V=Hom(\oplus H^{p,p}(X),\mathbb{C})$.
\end{theo}
\subsubsection*{Example}
Let $E$ be an elliptic curve. We can identify the moduli of elliptic curves with $$SL_2(\mathbb{Z})\backslash \mathrm{Stab}/\mathbb{C}^* $$
where the first quotient by $SL_2(\mathbb{Z})$ comes from Fourier-Mukai.

\bigskip

\subsubsection*{Motivating conjecture:}
Let $X$ be a smooth, projective Calabi-Yau threefold. If $Y_t\rightarrow T$  is the mirror, then $$\phi:T\longrightarrow Aut(\mathcal{D}^b(X))\backslash \mathrm{Stab}/\mathbb{C}^*, $$ where the map $\phi$ is defined in physics.


\subsubsection*{Example}
Let $S$ be a surface and $H$ an ample line bundle on it.
\begin{theo}(Bridgeland, Arcara-Bertram)
The function $$Z=-d+ir$$ supports a stability condition on $S$ if $\alpha_d=\alpha_0+\alpha_1+\alpha_2\in \tilde{NS}(S)$ satisfies $\alpha_1^2>2\alpha_0\alpha_2$, and $\alpha_0>0$.
\end{theo}

Suppose $d(E)=(Ch(E),\alpha_d)$, $r(E)=(Ch(E),\alpha_r)$ ($\alpha_r=\alpha_dH$), then $Z$ has a tilt $\sigma$ of $Coh(S)$.

\medskip
Consider $\mathcal{M}=\mathcal{M}^{\tiny{stable}}_S(c,(\sigma,Z))$, and as we vary $(\sigma,Z)$ we get birational models of $\mathcal{M}$.
\subsubsection*{Example} Let $S=\mathbb{P}^2$. We will interpret $\mathbf{Hilb}^n(\mathbb{P}^2)$ the Hilbert scheme of points on $\mathbb{P}^2$ as the moduli id ideal sheaves, $i.e.$ $(1,0,-n)$.
\begin{theo}(\cite{ABCH}) The birational models of $\mathbf{Hilb}^n(\mathbb{P}^2)$ that arise from divisors in the Mori chambers of the effective cone $\mathrm{Eff}(\mathbf{Hilb})$ correspond to moduli spaces of Bridgeland stable objects.
\end{theo}


\bigskip
\begin{thebibliography}{C}

\bibitem[ABCH]{ABCH}D. Arcara, A. Bertram, I. Coskun, J. Huizenga \emph{The minimal model program for the Hilbert scheme of points on the plane and Bridgeland stability}, Advances in Mathematics, Vol. 235, 2013.

\end{thebibliography}

\end{document}
